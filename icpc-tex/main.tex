\documentclass[landscape,twocolumn,9pt]{jsarticle}
\usepackage[top=10mm,bottom=10mm,left=10mm,right=10mm,headsep=1mm,footskip=6mm]{geometry}
\usepackage[utf8]{inputenc}
\usepackage{amsmath,amssymb}
\usepackage{listings,jlisting,color}
\usepackage{txfonts}

\lstset{
  language=C++,
  basicstyle=\footnotesize\ttfamily,
  directivestyle= \color[rgb]{0.40,0.40,0.40}\textbf, % #666666
  keywordstyle=   \color[rgb]{0.00,0.40,0.60}\textbf, % #006699
  keywordstyle=[2]\color[rgb]{0.40,0.00,0.60}\textbf, % #660099
  commentstyle=   \color[rgb]{0.00,0.51,0.00},        % #008200
  stringstyle=    \color[rgb]{0.00,0.00,1.00},        % #0000ff
  showstringspaces=false,
  frame=single,
  framexleftmargin=18pt,
  xleftmargin=18pt,
  numbers=left,
  numbersep=1em,
  basewidth={0.54em,0.45em},
  lineskip=-2pt,
  breaklines=true,
  breakindent=0pt,
  tabsize=2
}

\begin{document}

\section{misc}

\subsection{.bashrc}
\lstinputlisting{../misc/.bashrc}
\subsection{.vimrc}
\lstinputlisting{../misc/.vimrc}
\subsection{テンプレート}
\lstinputlisting{../misc/template.cpp}

\section{データ構造}

\subsection{Union Find Tree}
\lstinputlisting{../datastructure/unionfind.cpp}
\subsection{Segment Tree}
\lstinputlisting{../datastructure/segmenttree.cpp}
\subsection{Fenwick Tree}
\lstinputlisting{../datastructure/fenwick.cpp}
\subsection{Lazy Segment Tree}
\lstinputlisting{../datastructure/lazysegtree.cpp}
\subsection{Link-Cut Tree}
\lstinputlisting{../datastructure/linkcut.cpp}

\section{グラフ}

\lstinputlisting{../graph/graph.cpp}
\subsection{Dijkstra}
\lstinputlisting{../graph/dijkstra.cpp}
\subsection{Bellman-Ford}
\lstinputlisting{../graph/bellman.cpp}
\subsection{Kruskal}
\lstinputlisting{../graph/kruskal.cpp}
\subsection{関節点,二点連結成分}
\lstinputlisting{../graph/articulation.cpp}
\subsection{橋,二辺連結成分}
\lstinputlisting{../graph/bridge.cpp}
\subsection{トポロジカルソート}
\lstinputlisting{../graph/topologicalsort.cpp}
\subsection{強連結成分分解}
\lstinputlisting{../graph/scc.cpp}
\subsection{無向グラフの最小カット}
\lstinputlisting{../graph/maordering.cpp}
\subsection{Lowest Common Ancestor}
\lstinputlisting{../graph/lca.cpp}

\section{フロー}

\subsection{Ford-Fulkerson}
\lstinputlisting{../graph/fordfulkerson.cpp}
\subsection{Edmonds-Karp}
\lstinputlisting{../graph/edmondskarp.cpp}
\subsection{Dinitz}
\lstinputlisting{../graph/dinitz.cpp}

\section{文字列処理}

\subsection{接尾辞配列}
\lstinputlisting{../string/suffixarray.cpp}

\subsection{Aho-Corasick}
\lstinputlisting{../string/aho.cpp}

\section{計算幾何}

\subsection{幾何の基本要素}
\lstinputlisting{../geometry/geometry.cpp}

\subsection{交点計算}
\lstinputlisting{../geometry/crosspoint.cpp}

\subsection{接線の計算}
\lstinputlisting{../geometry/tangent.cpp}

\subsection{多角形の取り扱い}
\lstinputlisting{../geometry/polygon.cpp}

\subsection{凸包}
\lstinputlisting{../geometry/convexhull.cpp}

\subsection{凸多角形の切断}
\lstinputlisting{../geometry/convexcut.cpp}

\subsection{線分アレンジメント}
\lstinputlisting{../geometry/arrangement.cpp}

\subsection{最近点対}
\lstinputlisting{../geometry/closestpair.cpp}

\section{数理}

\subsection{FFT}
\lstinputlisting{../math/fft.cpp}

\end{document}
